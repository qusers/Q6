%
% PACKAGES
%
\documentclass[a4paper,12pt]{article}
\usepackage{graphics}
\usepackage[dvips]{graphicx}
\usepackage{times}\fontfamily{ptm}\selectfont
\usepackage[activeacute,english]{babel}
\usepackage{t1enc}
\usepackage{amsmath}

\setlength{\textheight}{9.1in} \setlength{\textwidth}{6.0in}
\setlength{\topmargin}{-0.18in} \setlength{\oddsidemargin}{0in}
\pagestyle{plain} \pagenumbering{arabic}
\newcommand{\HR}{\rule{1em}{.4pt}}
\newcommand{\sctn}[1]{\section{\large #1}}
\newcommand{\subsctn}[1]{\subsection{#1}}
\newcommand{\subsubsctn}[1]{\subsubsection{#1}}
\renewcommand{\theenumi}{\alph{enumi}}
\renewcommand{\theenumii}{\arabic{enumii}}
\newcommand{\dGb}{\ensuremath{\Delta G_{\rm bind}}}
\newcommand{\dGel}{\ensuremath{\Delta G_{\rm el}}}
\newcommand{\dGvdw}{\ensuremath{\Delta G_{\rm vdW}}}
\newcommand{\qdyn}{\texttt{Qdyn5}}
\newcommand{\qprep}{\texttt{Qprep5}}
\newcommand{\qcalc}{\texttt{Qcalc5}}
\newcommand{\q}{\texttt{Q}}
\newcommand{\grep}{\texttt{grep}}
\newcommand{\sed}{\texttt{sed}}
\newcommand{\runsh}{\texttt{run{\_}Q.sh}}
\newcommand{\pymol}{\texttt{pymol}}

\author{Martin Alml�f, Martin And\'er, Sinisa Bjelic, Jens Carlsson, \\
Hugo Guti\'errez de Ter\'an, Martin Nervall, Stefan Trobro and
Fredrik �sterberg}
\date{18th August 2005}

%
% DOCUMENT
%

\title{3. Binding affinity estimation using the Linear Interaction Energy method}

\begin{document}

\maketitle
\tableofcontents
\newpage


\sctn{Introduction}

In this practical the free energy of binding will be calculated
for two ligands, camphor (CAM) and camphane (CMA), to the P450cam
receptor using molecular dynamics (MD) together with the Linear
Interaction Energy (LIE) method. The example is taken from the
paper Alml�f {\it et al}, J. Comput. Chem., 25, 1242-1254,(2004),
but with some simplifications and automation
of the steps.\\

\subsctn{Linear Interaction Energy method} The LIE method for the
calculation of the {\emph {absolute}} free energy of binding
({\dGb}) was first proposed by �qvist {\it et al} in 1994. It is a
semi-empirical method, which is computationally less expensive
than the rigorous free energy perturbation (FEP) method used in
the previous practical. The LIE approach is similar to FEP, but
the method only requires simulations of the free and bound
ligand, {\it i.e.} it does not involve any transformation process
as in FEP. LIE also overcomes some of the difficulties associated
with a FEP calculation, such as convergence. The estimated free
energy of binding is calculated as a linear combination of the
differences in the average ligand-surrounding (l-s) interactions
in water and protein. Interaction energies are split into
electrostatic ({\it el}) and van der Waals ({\it vdw}) terms, and
weighted by different factors

\begin {equation}
\Delta G_{\rm bind} = \alpha (\langle V^{vdw}_{l-s}\rangle _{\rm
p}
              - \langle V^{vdw}_{l-s}\rangle _{\rm w})
              + \beta (\langle V^{el}_{l-s}\rangle  _{\rm p}
              - \langle V^{el}_{l-s}\rangle  _{\rm w})
              + \gamma
\label{eq_lie}
\end {equation}

\noindent where the brackets denote thermal averages sampled
during the MD simulations of the protein (p) and water (w)
environments. The main idea of the method is to consider polar and
non-polar contributions to the free energy separately.

\begin {itemize}
\item {\bf The polar contribution}, \ensuremath{\beta \Delta
\langle V^{el}_{l-s}\rangle}: The scaling factor
\ensuremath{\beta}=0.5 is theoretically derived from electrostatic
linear response theory and yields very good agreement with
experimental solvation energies for ionic solutes. For uncharged
compounds, FEP calculations have shown that lower
\ensuremath{\beta} values are necessary and can be assigned by a
simple scheme depending on the ligand's chemical nature. For the
ligands considered here \ensuremath{\beta}=0.43 (group: neutral
ligand with no hydroxyl groups). \item {\bf The non polar
contribution}, \ensuremath {\alpha \Delta \langle
V^{vdw}_{l-s}\rangle}: The non-polar contributions to the free
energy of binding are assumed to have a linear relationship with
the surrounding van der Waals energies. The scaling factor,
\ensuremath{\alpha}, is empirical, but a value of 0.18 has worked
well for a large number of systems, including the system
considered here. \\ \\ \noindent An additional constant term,
\ensuremath{\gamma}, may be required to set the absolute binding
free energy scale and it is dependent on the hydrophobicity of the
binding site. That is, it is specific for a given protein but does not affect
relative binding free energies. For
P450cam the optimal value of this constant term is
$\gamma=-4.5$ kcal/mol.
\end {itemize}

\newpage

%
%%%%%%%%%%%% Methods (day 1)%%%%%%%%%%%%%%%%%%%%%%%%%%%%%%%%%%%%%%%%
%
\sctn{Methods: Prepare and run the MD simulations} Two separate MD
simulations of the ligand ("water" and "protein" simulations) have
to be carried out to get an estimate of the free energy of binding
using LIE.

%
%%%%%%%%%%%%%%%%%%%%%%%%%%%%%%%%%%%% Protein simulation %%%%%%%%%%%%%%%%%%%%%%%%%%%%%%%%%%%
%
\subsctn {Protein simulation}

Go to the directory named "LIE/protein", where one directory for
each of the ligands ("CAM" and "CMA") has been created. In each of
these you will find

\begin {itemize}
\item The coordinate file ({\texttt {complex.pdb}}) prepared as
described in the first practical. For CAM and CMA the coordinates
were taken from the files with PDB codes 6CPP and 2CCP.

\item The topology {\texttt {lig.top}} generated by {\qprep} as
shown in the first practical, the input files {\texttt
{name.inp}}, and a FEP file {\texttt {lig.fep}}.

\begin {itemize}
 \item Go to the directory named "CAM". Type
{\texttt {pymol complex.pse}} in your terminal window to see the
structure of the protein-ligand complex. Repeat this procedure for
"CMA". \item Compare the FEP file for LIE to the one used in the
free energy perturbation practical. What are the differences
between the files? The ligand-surrounding energies will be
evaluated for the group of atoms specified in the FEP file. These
are also called "Q atoms" by the program.
\end {itemize}


\end {itemize}

\noindent The MD simulation is divided into two phases:
    \begin {itemize}
    \item {\bf Equilibration phase}: The equilibration phase consists of stepwise heating from 1 to 300 K, with heavy solute atoms restrained
    to their crystallographic positions. The equilibration scheme
    is divided into five steps {\texttt {eq1.inp}} to {\texttt {eq5.inp}} that are similar to that used in the first practical.
    \begin {itemize}
    \item Compare {\texttt {eq5.inp}} for the protein simulations of CAM and CMA. There is one difference between these files. What? (HINT: For the CMA ligand there is a conserved water bound to the
    HEME group, but not for CAM)
     \end {itemize}
 \end {itemize}

   \begin {itemize}
    \item {\bf Production phase}: The MD simulation is divided
    into 15 consecutive blocks named {\texttt {md1.inp}} to {\texttt {md15.inp}}.
     \begin {itemize}
    \item How long is the total simulation (in ps)?
    \end {itemize}
    \end {itemize}
\noindent Running the MD simulations would take a few hours. In
order to save time the output files have been prepared in the
directory "results" for each ligand.

%
%%%%%%%%%%%%%%%%%%%%%%%%%%%%%%%%%%%% Water simulation %%%%%%%%%%%%%%%%%%%%%%%%%%%%%%%%%%%
%
\subsctn {Water simulation} Now move to the directory "LIE/water",
where one directory for each ligand has been created. In these you
will find

\begin {itemize}
\item The coordinates of the ligand {\texttt {ligand.pdb}}
extracted from the original PBD files, but with all the protein
atoms removed.

\item The topology generated by {\qprep} with the ligand
positioned in the center of a sphere of radius 18� filled with
water molecules, the input files, and a {\texttt {FEP}} file. All
files are named according to the same convention as for the
protein simulation.

\begin {itemize}
\item Move to the "CAM" or "CMA" directory. Type {\texttt {pymol
ligand.pse}} in your terminal window. The ligand is positioned in
the center of the sphere and surrounded by water.
\end {itemize}

\end {itemize}

\noindent As previously the MD simulation is divided into two
phases.
    \begin {itemize}
    \item {\bf Equilibration phase}: The equilibration phase ({\texttt {eq1.inp}} to {\texttt {eq5.inp}}) is
    similar to the protein simulation, but one important change
    has been made. Since there is no protein present in the simulation, only a restraint that keeps the center of mass of the ligand in the sphere center has been added: \\ \\
    {\texttt {[sequence{\_}restraints]\\
    1 27 10 1 2}}  \\
    \item {\bf Production phase}: The production phase is basically the same as with the complex. The restraint that keeps the ligand in the center of the sphere is maintained to ensure a correct solvation of the ligand. \\
    \end {itemize}

\noindent Again, running the MD simulations would take a few
hours. In order to save time the output files have been prepared
in the directory "results" for each ligand.

\newpage
%
%%%%%%%%%%%% Results (day 2)%%%%%%%%%%%%%%%%%%%%%%%%%%%%%%%%%%%%%%%%
%
\sctn {Results: Evaluating the simulations} Now we will evaluate
energies from the MD simulations, the conformations of the system
and specific ligand-protein interactions. \subsctn {Analysis of
P450cam binding of CAM and CMA}

\begin {itemize}
\item {\bf Convergence: }First we will look at the convergence and
results of the ligand-surrounding energies for the ligands in
water and protein. By using the {\texttt {perl}} script {\texttt
{energies.pl}}, the {\it electrostatic} and {\it van der Waals}
energies are extracted from the {\texttt {name.log}} files.

\begin {itemize}
\item Move to "protein/CAM/results" directory. \item Open a
{\texttt {name.log}} file and try to find the ligand-surrounding
energies for the water and protein simulations. \item Now run the
script {\texttt {perl energies.pl md 1 15}}, which extracts,
summarizes and plots all ligand-surrounding energies for the
entire production phase ({\texttt {md1.log}} to {\texttt
{md15.log}}).

\begin {itemize}
\item Are there any large changes in the energies throughout the
simulations?
 \item The script has also calculated the difference
between the average energies for the first and second halves of
the simulation. This can be considered as a measure of the
convergence error. Have the simulations converged?
\end {itemize}
\item Repeat this procedure for all simulations and summarize the
ligand-surrounding energies in Table 1.
\end {itemize}

\begin{table} [htb1]
\centering
\begin{tabular} {|l||c|c|c|c|}
 \hline
 Ligand & \ensuremath {\langle V^{vdw}_{l-s}\rangle _{\rm p}} & \ensuremath {\langle V^{vdw}_{l-s}\rangle _{\rm w}} & \ensuremath {\langle V^{el}_{l-s}\rangle _{\rm p}} & \ensuremath {\langle V^{el}_{l-s}\rangle _{\rm w}} \\
 \hline
 \hline
  CAM  &  &  & &\\
 \hline
  CMA  &  &  & &\\
 \hline
 \end{tabular}
 \caption{Ligand-surrounding energies for CAM and CMA }
 \end{table}

\newpage

\item {\bf Binding free energies:} The calculated
ligands-surrounding energies can be used to estimate the free
energy of binding.

\begin {itemize}
\item Calculate the LIE binding free energies using the
parameterization; \ensuremath{\alpha}=0.18,
\ensuremath{\beta}=0.43 and \ensuremath{\gamma} = -4.5 and
summarize these into Table 2. \item Are the calculated binding
free energies in agreement with experiment? Are the ligands
correctly ranked? \item Is the binding due to electrostatic or
hydrophobic interactions?
\end {itemize}

\begin{table} [htb2]
\centering
\begin{tabular} {|l||c|c|c|c|}
 \hline
 Ligand & \ensuremath {\Delta \langle V^{vdw}_{l-s}\rangle} & \ensuremath {\Delta \langle V^{el}_{l-s}\rangle} & \ensuremath {\Delta G_{bind,calc}} & \ensuremath {\Delta G_{bind,exp}}  \\
 \hline
 \hline
  CAM  &  &  &  & -7.90\\
 \hline
  CMA  &  &  &  & -5.91\\
\hline
 \end{tabular}
 \caption{ Differences in ligand-surrounding energies and calculated free energies using LIE}
 \end{table}

\item {\bf Structures:} Viewing the structures after the different
parts of the simulation is a very important part of the evaluation
of the simulations.

\begin {itemize}
\item Move to a "results" directory. \item To look at how the
ligand moves as a function of time, we will use a {\pymol} script.
Type {\texttt {pymol simulation.pse}} in your terminal window and
the program will load about 150 snapshots from the simulation.
Start the simulation by pressing play in the lower-right corner.

\item Calculate the average structures from the MD simulation
using {\qprep} for each protein simulation. Type {\texttt
{{\qprep} < qprep{\_}av1.inp > qprep{\_}av1.out}} and then
{\texttt {{\qprep} < qprep{\_}av2.inp > qprep{\_}av2.out}} in your
command window to get the average structures of the first and
second halves of the simulations. Use {\pymol} to compare the
these two pdb files ({\texttt {average{\_}md1to7.pdb}} and
{\texttt {average{\_}md8to15.pdb}}). Are the simulations
structurally converged? Also compare the structures to the crystal
structure ({\texttt {masked.pdb}}). Do the simulations agree with
the crystal structure?

Here is a list of useful {\pymol} tips

    \begin {itemize}
    \item Load pdb file: type {\texttt{load file.pdb}} for all the files you want to
    load.
    \item Mouse control: left button rotates, middle button translates, right button zooms.
    \item Selections: for selecting the ligand, type {\texttt{sel lig, res 407}} and the residue 407 will be the selection "lig". You can change colors or representation on it. See the list of selections on the right side of the interface, and explore all the possibilities
    \end {itemize}

\end {itemize}

\newpage

\item {\bf Key interactions:} To elucidate which residues
contribute most to binding, the energetic interactions of the
ligand with its surrounding groups can be calculated. This has
been carried out using the program {\qcalc}.

\begin {itemize}
\item Go to the "protein/CAM/results" directory. \item The
residue-ligand interactions for residue 1 to 406 to the ligand
(residue 407) for all production files has been saved to the file
{\texttt{res{\_}lig.txt}}. Type {\texttt{gnuplot}} in your
terminal window. Now type {\texttt {load 'residue.plt'}} to plot
the residue-ligand interactions. Can you identify the residues
that play an active role in ligand binding?

\item Repeat the same procedure for CMA.

\end {itemize}
\end {itemize}

\end{document}
